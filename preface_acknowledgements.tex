\prefacesection{Acknowledgements}

The idea to pursue graduate school chrystalized when I was working as a quant for Teza Technologies, a small, academic high frequency trading firm. I witnessed first hand how machine learning and low latency technologies were completely up-ending securities trading. Broad shouldered Wall Street suites were being replaced by hackers and math nerds. As machine learning and automation became increasingly powerful tools for our firm, I began to question whether these tools could spur similar advances in physics and cosmology. 

In 2016, I started the Institute for Computational and Mathematical Engineering (ICME) MS program and began working with Philip (Phil) Marshal on high dimensional cosmological inference. In the summer of 2017, Phil introduced me to my eventual advisor, Steven (Steve) Kahn, as someone who could potentially support me for a PhD. I remember our first meeting well. I presented my results. Steve sat there, completely silent. His stern expression gave no indications of his thoughts ... was he even paying attention? After I finished presenting, he asked extremely insightful questions that demonstrated he had been paying careful attention. Indeed, Steve is a great listener. I was also impressed by his creativity and interest in working on problems off the beaten path. Shortly after our first encounter we teamed up to work on star trails, and Steve officially became my PhD advisor. 

Over the past four years we focused primarily on subsecond photometry and wavefront sensing. In both cases, after experimenting with and attempting to extend existing approaches, we were able to develop superior approaches based on machine learning. While these solutions seem very natural in hindsight, it is worth reminding any future PhD students reading this, that we proceeded through many many failed attempts at tackling these problems, received a lot of skepticism around leveraging deep learning initially, before we were finally able to make it work. As Edison says ``I just found 2,000 ways not to make a lightbulb; I only needed to find one way to make it work.'' In this thesis, we focus on our work on wavefront sensing. We point readers interested in our work on star trail photometry to our published and forthcoming papers.

I also would like to thank Patricia Burchat and Aaron Roodman for their guidance and support. As a student from ICME, there were times where I did not feel welcome by the broader physics community. However, Professors Burchat and Roodman always were extremely welcoming and kind. Professor Burchat has been supportive throughout my PhD and provided key feedback to my research efforts and this thesis. Professor Roodman, who developed a previous generation wavefront sensing system, was also an extremely instrumental mentor. My failed attempts to extend his Fourier transform and optimization based method gave me a clearer sense of the key challenges for Rubin and ultimately allowed me to develop a new approach to tackle them. Both of these Professors also asked great questions during my oral defense.

Professor Stephen Boyd was another crucial member of my defense committee. I enjoyed working with Professor Boyd to map the covariance estimation problem for the global wavefront into a convex optimization problem, and on optimal control. This collaboration was a great example of inter-disciplinary research. While Professor Boyd no longer teaches his infamous CME 364A course, Convex Optimization, his recorded 2008 lectures on YouTube have over one million views. These are largely credited as popularizing the subject, and helped to make this my favorite course at Stanford. I believe it provides an unrivalled foundation for machine learning and statistical learning, and encourage any readers to take it - or at least watch Boyd's lectures. 

I would also like to thank my collaborators. Joshua Meyers has been key a mentor and go-to expert on all things optics. His Batoid Python package significantly lowered the hurdles to doing raytracing and really enabled this work. I have also tried to pick up some Joshism's, such as his clean code and notebooks, tendency to use widgets, and clear presentations paced with incredible simulations. 

Federica Bianco has also been a key collaborator on the star trails work. Federica's enthusiasm for science, which ranges from astrophysics to remote sensing to climate change to extraterrestrial technosignatures - is extremely inspiring. She was always willing to meet our projects where they were, and patiently brainstorm and work through the obstacles that come with trying to break new ground. 

I am also extremely grateful to my collaborators at the National Optical Astronomy Observatory in Tucson. Steve helped arange for me to move out there for two quarters in my third year to collaborate with the team and learn the Rubin (then LSST) software stack and wavefront sensing building blocks. The background required for wavefront sensing is not taught in a traditional physics program. In fact, I have still not come across a good \textit{single} reference. Instead one's knowledge must be gleaned from many disparate sources and through experience. Thus, I am very appreciative of the time Bo Xin and Te-Wei Tsai spent with me to impart this wisdom. I also appreciated collaborating with Sandrine Thomas and Chuck Claver on various challenges of the baseline approach the team was developing for Rubin. 

The ICME also played a big role in my development. An ICME graduate can employ more mathematical rigor than most engineers and scientists, and also has strong coding skills. I can't think of better preparation to contribute to innovation in the 21st century. The Directors Margot Gerritsen and Gianluca Laccarino did a great job improving the program throughout my tenure. The former administrative assistant Indira Choudhury was also the spoke that made the institute run, and students graduate. I am forever grateful for her help in this endeavor. 

It has been an incredible priviledge to attend Stanford University. The school has top faculty, a beautiful campus, and ample support and extracurricular activities. It also in the heart of Silicon Valley and can serve as a stepping stone into entrepreneurship. Palo Alto is also a great starting point to explore the broader San Francisco Bay Area. Nowhere else can one surf Pacifica in the morning and meet with venture capitalists on Sand Hill Road in the evening.

I am also very appreciative of Stanford's PhD minor programs. In addition to my PhD in Computational and Mathematical Engineering I was also able to complete a PhD Minor in Physics and PhD Minor in Computer Science. This background has prepared me well to contribute to interfaces between fields where much opportunity remains. I am the first student at Stanford to complete these degrees. 

The past three summers I was able to augment my PhD with three formative experiences. In the summer of 2019, I mentored Stanford undergraduate Emily Li. Emily was a bright and enthusiastic young researcher. She absorbed the optics for wavefront sensing like a sponge. This experience taught me how to clearly communicate expectations, share the joy of science, and teach foundational skills. The experience encouraged her to pursue a career in physics. Nothing is more rewarding than seeing one's mentees succeed.  

In summer 2020, I interned as a Quantitative Researcher in the Global Quantitative Strategies business at Citadel Investment Group. It was great to see how state of the art quantitative hedge funds operate and compare and contrast the methods used here with those used during my time in high frequency trading. While I cannot go into detail about my work there due to a strict non-disclosure agreement, I can say that there is a lot of value in being fluent in the mathematics behind statistical learning models as well as being a strong programmer and knowing how to accelerate and improve these models and their training on modern compute infrastructure. Again, the ICME combination of mathematical rigor and computational engineering was superb preparation. This experience was also a great reminder of how challenging it is to beat the market. Even with a team of hundreds of highly accomplished PhD researchers, state of the art software and resources, and the accumulation of millions of signals and research findings ... it is difficult to break even. I suspect people who have not experienced this world first hand, and the repeated failures that accompany this amibitious goal, with billions of dollars on the line, cannot truly appreciate how efficient markets really are. Retail trading and investing is pure speculation - and a losing game. 

In summer 2021, I interned as a Machine Learning Engineer on the Instagram Ads Ranking and Delivery team at Facebook. I really enjoyed researching and developing large scale recommender systems. These systems, used in everything from Google's search engine to prioritizing content on Facebook's home feed, are some of the most critical components of large tech businesses and the modern internet. The computational resources and datasets available at companies like Facebook are so superior to those in academia that to perform cutting edge work in this field almost necesitates one to work for a large tech company. Some of my colleagues were former faculty members who moved from academia to research roles at Facebook in order to get around this discrepancy and continue producing impactful research. It seems like it will be increasingly difficult for academia to retain top talent in these fields: large scale recommender systems and machine learning more broadly.  

My family has always been incredibly supportive. My father David \textit{Evan} Thomas served in the United States Navy for 25 years. I am incredibly proud of his committment and dedication to his country. When I was growing up, we would stay with relatives over the holidays so he could work over his vacations to earn enough money to send my sister to a private high school. That was a good investment, because today my sister Danielle Thomas has become a very successful orthopedic surgeon. She is a tough cookie - her favorite tool is the bone saw - and has no problem exceling in a field that is over 90 percent male. She also is a voracious reader. She read over one hundred books in each of the last two years. My mother Valerie Thomas was my original teacher during infancy and has encouraged my pursuits ever since. During my PhD, she would regularly visit the bay area and would send me care packages. She has become an incredible painter. Her paintings decorate my walls; guests tend to be surprised that the artist is, in fact, my mother.

Finally, I would like to give the utmost thanks to my wife Madeleine Scott. We met as undergraduates at MIT and started dating in 2012. She briefly worked at a biotech startup before pursing both an MD and a PhD at Stanford through the eight year, fully funded, and \textit{extremely} competitive Medical Scientist Training Program. She is an incredibly kind person that feels so much for other people that she cannot watch dramatic movies. She loves helping others solve problems and will be an incredible physician. She is also the most talented young scientist that I know. This past year she has had two papers accepted to the revered journal Nature. I am very fortunate to be her collaborator in life. 

My wife's parents Frances Donovan and Gary Scott have been incredibly supportive throughout our relationship and especially during the Covid-19 pandemic. During Covid we have stayed with them in Berkeley for months at a time. Their active lives, copious reading, fascinating conversations, and respect for each other set a great example. While, Gary, a Physics PhD who now works as a molecular biology researcher, enjoys shooting down dark matter, lambda-CDM, and other tennets of modern cosmology - I cannot thank him and Frances enough for letting me marry their daughter. 

Recently, we welcomed a new member into our family. Kyver Blaze Thomas was born at 3:48 pm on July 24th at the Lucile Packard Children's Hospital. We are absolutely smitten new parents. Kyver is already good-natured like his mom, and inquisitive like his dad. Perhaps one day he will read this thesis.
